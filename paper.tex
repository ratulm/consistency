
\newif\iflongversion
% To use shorter version, comment out this line...
%\longversiontrue
% and uncomment this line:
\longversionfalse

%\documentclass{sig-alt-hotnetscr}
\documentclass{sig-alternate-10pt-2013}
\usepackage[usenames,dvipsnames]{color}
\definecolor{mycitegreen}{RGB}{0,90,0}
\definecolor{myrefblue}{RGB}{90,0,0}
\definecolor{cyred}{RGB}{140,0,0}
\definecolor{commentgreen}{RGB}{0,80,0}
\usepackage[draft=true, final=true, colorlinks=true, citecolor=mycitegreen, linkcolor=myrefblue, anchorcolor=myrefblue, pagebackref=false]{hyperref}
\usepackage{graphics,enumitem}
\usepackage{epsfig}
\usepackage{graphicx}
%\usepackage{floatrow}
\usepackage[font={bf,small}]{caption}
\usepackage[captionskip=0pt, font={small,bf}, farskip=2pt]{subfig}
\usepackage{multirow}
\usepackage{array}
\usepackage{color}
\usepackage{url}
\usepackage{xspace}
\let\proof\relax
\let\endproof\relax
\usepackage{amsthm}
\usepackage{boxedminipage}
\usepackage{colortbl}
\usepackage{amsmath}
\usepackage{amsfonts}
\usepackage{amssymb}
\usepackage{mathrsfs}
\usepackage{graphicx}
% \usepackage{natbib}
\usepackage{subfig}
\usepackage{setspace}

% \setlength{\bibsep}{2pt}

%\usepackage[charter]{template/mathdesign/mathdesign}
%\usepackage{MinionPro}
%\setlength{\textheight}{9.4in}
%setlength{\textheight}{9.25in}
%\setlength{\textwidth}{6.75in}
%\setlength{\columnsep}{.33in}
%\usepackage[]{geometry}

% Compact itemize and enumerate.  Note that they use the same counters and
% symbols as the usual itemize and enumerate environments.
\def\compactify{\itemsep=0pt \topsep=0pt \partopsep=0pt \parsep=0pt}
 \let\latexusecounter=\usecounter
 \newenvironment{CompactItemize}
   {\def\usecounter{\compactify\latexusecounter}
    \begin{itemize}}
   {\end{itemize}\let\usecounter=\latexusecounter}
 \newenvironment{CompactEnumerate}
   {\def\usecounter{\compactify\latexusecounter}
    \begin{enumerate}}
   {\end{enumerate}\let\usecounter=\latexusecounter}

%\baselineskip=12.1bp
%\showthe\baselineskip

\setlength{\pdfpagewidth}{8.5in}
\setlength{\pdfpageheight}{11in}

\makeatletter
\newif\if@restonecol
\makeatother
\let\algorithm\relax
\let\endalgorithm\relax
\usepackage[figure,vlined]{algorithm2e}

\DeclareMathOperator*{\argmax}{arg\,max}

\title{On Consistent Updates in Software Defined Networks 
\iflongversion
(Extended Version)
\fi
}

\numberofauthors{2}
\author{
\begin{tabular}{cc}
Ratul Mahajan
&
Roger Wattenhofer
\\
\affaddr{Microsoft Research}
&
\affaddr{ETH Zurich}
\end{tabular}
}

\begin{document}

\iflongversion
\else
\conferenceinfo{Hotnets '13,} {November 21--22, 2013, College Park, MD, USA.} \CopyrightYear{2013} \crdata{978-1-4503-2596-7}
\fi

\clubpenalty=10000
\widowpenalty = 10000

\maketitle
%\baselineskip=12bp

% Paragraph Highlight
\newcommand{\paragraphb}[1]{\vspace{0.03in}\noindent{\bf #1} }
\newcommand{\paragraphe}[1]{\vspace{0.03in}\noindent{\em #1} }
\newcommand{\paragraphbe}[1]{\vspace{0.03in}\noindent{\bf \em #1} }

% Comment colors
%\newcommand{\comment}{\textcolor{black}}
\newcommand{\prooffontsize}{\fontsize{8pt}{9.3pt}\selectfont}
\newcommand{\cready}{\textcolor{black}}
\newcommand{\newcomment}{\textcolor{black}}

\newcommand{\ratul}[1]{{\color{blue}{#1}}}
\newcommand{\roger}[1]{{\color{green}{#1}}}

\newcommand{\swan}{\normalsize S{\small WAN}\xspace}
\newcommand{\swansmall}{\small S{\scriptsize WAN}\xspace}
\newcommand{\cflow}{Vflow\xspace}
\newcommand{\flow}{vflow\xspace}
\newcommand{\flows}{vflows\xspace}

\newcommand{\topolarge}{\textsf{IDN}\xspace}
\newcommand{\toposmall}{\textsf{G-Scale}\xspace}

\newcommand{\corule}{Rule\xspace}
\newcommand{\orule}{rule\xspace}
\newcommand{\orules}{rules\xspace}

\newcommand{\cpr}{Service\xspace}
\newcommand{\pr}{service\xspace}

\newcommand{\cprops}{Service properties\xspace}
\newcommand{\props}{service properties\xspace}

\newcommand{\trurl}{\url{http://somewhere}}

% To make the FIXMEs go away, comment out this line...
\newcommand{\fixme}[1]{{\bf\textcolor{red}{[#1]}}}
% ...and uncomment this one.
%\newcommand{\fixme}[1]{}

\newcommand{\helpme}[1]{{\bf\textcolor{red}{#1}}}

\newcommand{\figtocapskip}{\vspace{-6pt}}
\newcommand{\figtocapbigskip}{\vspace{-9pt}}
%\newcommand{\interfigskip}{\vspace{-12pt}}
%\newcommand{\listskip}{\vspace{-5pt}}
%\newcommand{\listskips}{\vspace{-3pt}}
%\newcommand{\algindent}{\hspace{13pt}}

\theoremstyle{definition}
\newtheorem{lemma}{Lemma}


\begin{sloppypar}
\textbf{Abstract---}
We argue that to fully realize the promise of SDNs we need robust methods to update the data plane state in a way that maintains desired consistency properties such as packets should not be dropped. We highlight the inherent trade-off between the strength of the consistency property and the dependencies it imposes among rules at different switches.
For one basic consistency property---loop freedom---we develop an update algorithm that has provably minimal dependency structure.
We sketch a general architecture for consistent network updates which separates the twin concerns of correctness and efficiency. 


\iflongversion
\else
\begin{small}
\vspace{3pt}
\paragraphb{Categories and Subject Descriptors:} C.2.1 [Computer-Communication Networks]: Network Architecture and Design
\vspace{-2pt}
\paragraphb{General Terms:} Algorithms
%\vspace{-2pt}
%\paragraphb{Keywords:} Inter-DC WAN; software-defined networking
\end{small}
\fi

\section{Introduction}
\label{sec:intro}

Software-defined networking (SDN), as envisioned today in separation of control and data planes and centralized control plane computation, is a relatively new concept.
%receiving a lot of attention in the research and practitioner communities.
From early papers (e.g.,~\cite{rcp-case,rcp,4d,ethane}), we learn that its primary promises were $i)$ centralized control plane computation can eliminate the ill-effects (e.g., looping packets) of distributed computation, and $ii)$ separation of control and data planes enables flexible configuration of the data plane in a manner that satisfies a wide range of policy concerns.
For example, the Routing Control Platform (RCP)~\cite{rcp-case,rcp} was motivated by ``oscillations" and ``loops" that can occur in certain iBGP architectures. To avoid these problems, it advocated a centralized control plane that directly configured the data plane of routers in an autonomous system.

4D aimed to simplify network management~\cite{4d}. It observed that the ``data plane needs to implement, in addition to next-hop forwarding, functions such as tunneling, access control, address translation, and queuing." In today's networks, this ``requires complex arrangements of commands to tag routes, filter routes, and configure multiple interacting routing processes, all the while ensuring that no router is asked to handle more routes and packet filters than it has resources to cope with. A change to any one part of the configuration can easily break other parts." Based on this observation, it argues for centrally computing data plane state in way that obeys all concerns.

Similarly, ETHANE argued that for simplified management of enterprise networks ``policy should determine the path that packets follow"~\cite{ethane}. It then argued for SDNs because the requirements of network management ``are complex and require strong consistency, making it quite hard to compute in a distributed manner." These promises have led to SDNs garnering a lot of attention from the researchers as well as practitioners.

However, as we gain more experience with this paradigm, a nuanced story is emerging.  Researchers have shown that, even with SDNs, packets can take paths that do not comply with policy~\cite{safeupdate,xx} and that more traffic can arrive at a link than its capacity~\cite{swan}. So, what explains this gap between the promise and these inconsistencies. The root cause is that promises apply to the eventual behavior of the network, {\em after} data plane state has been changed and the inconsistencies emerge {\em during} data plane state changes.

Since changes to data plane state, in response to failures, load changes, and policy changes, are an essential part of an operational network, so will be the inconsistencies. Thus, successfully using SDNs means that we need not only methods for computing policy-compliant data plane state but also methods to manage state changes in a way that maintains desired consistency properties.

Beyond the specific contradictory behaviors and solutions studies by prior work~\cite{safeupdate,swan}, we present a holistic analysis of such behaviors and what is fundamentally needed to prevent them.

We do not mean to imply that SDNs are not a net positive towards tackling the problems they aim to tackle. We merely wish to highlight that the story (i.e., use centralized computation) is not that straightforward.


5.	First, we outline a spectrum of consistency properties:
a.	Drop-freedom
b.	Loop-freedom
c.	Per-pkt consistency
d.	Capacity consistency
e.	…..

6.	Consistently migrating requires a careful planning of the order in which rules can be installed in the network. This means a dependence in which certain rules should be applied only after certain other rules have been applied.  Based on the observation, we outline a spectrum of the types of possible dependencies:
a.	None
b.	Downstream of the path
c.	Entire network

7.	Greater dependence slows transition. But certain consistency properties inherently require a minimum degree of dependence. We fill out the table …

8.	We propose new transition algorithms along the way

%\section{Background}
\label{sec:background}

Judging from the early papers on this topic, primary motivations for SDNs were to counter the ill-effects of distributed computation with partial knowledge and management complexity. distributed interactions between routers ...

The primary motivations 

For example, consider the Routing Control Platform (RCP)~\cite{rcp-case,rcp}, one of the first proposals that separated the control and data plane of routers. It used a centralized control plane that directly configured the routers in an autonomous system (AS). The primary motivations for this organization were "problems such as protocol oscillations and persistent loops"~\cite{rcp} that can occur in certain iBGP architectures.

\section{Loop-Free}
\label{sec:example}

\subsection{Example}

\begin{figure}[t!]
\label{fig:example}
\includegraphics[width=3in]{figures/noloops.png}
\caption{Running example, discussed in Section \ref{sec:example}.}
\end{figure}

Applying routing updates in a way that guarantees no loops is a basic but already interesting consistency property. Consider the five node example network with a single destination $d$ in Figure \ref{fig:example}. The existing routing to destination $d$ on the left of the figure should be updated according to the right of the figure. A naive way to update is to simply send out the forwarding updates (e.g., ``node $u$: for destination $d$ send to $x$''). However, when doing so, it might happen that node $x$ updates its rule before node $y$, introducing a routing loop between nodes $x$ and $y$. This loop will eventually disappear, namely once node $y$ updates its rule, but in an asynchronous system it is difficult to give guarantees when this will happen.

At the other end of the spectrum, the SDN controller could first (i) distribute the new rules, then (ii) wait for an acknowledgement of all the nodes that they have received the new rules, then (iii) tell all the nodes to stop sending packets, and finally, once (iv) they all acknowledged that, (v) tell the nodes to now use the new rules. After the nodes (vi) acknowledge that they are using the new rules, the SDN controller can tell them to (vii) remove the old rules, as they are not needed anymore. This solution does not suffer from loops, as the old and the new solution are well-separated in time.\footnote{Strictly speaking, this protocol is not sufficient, as there might be packets in transit, still using the old rules when rushing through the protocol. Consider a packet sent from sent from $y$ to $x$ shortly before $y$ received the order to stop sending (iii). If node $x$ already passed the last step of the protocol when receiving the packet, node $x$ has not other option then sending the packet back to node $y$, i.e. the packet is experiencing a loop.} However, it is also terribly slow. One may speed up the process by omitting steps (iii) and (iv). Now, assuming that node $x$ received the command to use the new rule (v) earlier than node $y$. As such, in order to guarantee no loop between nodes $x$ and $y$, we must introduce version numbers in packets such that nodes $x$ and $y$ know that packets from $x$ respectively $y$ must be treated according to the new respectively old rules. This is the solution proposed by \ref{theoriginalprincetonpaper}.

One may ask whether version numbers are really necessary, just to guarantee no loops? Also, one may ask whether a faster solution is possible. What nodes are really dependent on each other? This may look like a technicality, but as we all know, nodes are often temporarily unavailable or reacting slowly. If a node must wait for another node, there should better be a consistency reason for it, and not merely protocol overhead. Is there something like a \emph{minimal protocol for a given consistency guarantee}? This is exactly the question we address in this paper.

Regarding our example, the answer to this question is quite simple. Node $u$ does not even need to change its rule, as it always just forwards to $x$, hence node $u$ does not even have to be informed about the rule changes. Node $v$ can switch immediately after being informed about the new rule, as no matter whether using the old or new rule, the packet will always end up at node $x$, with no possibility to experience a loop. Also node $y$ can switch immediately to the new rule, as its packet will then directly reach the destination node $d$. The only critical node in our example is node $x$, for which we understand that node $x$ must wait until node $y$ implemented the switch, as otherwise our network might experience a loop.

More generally, achieving loop-free consistency requires rule updates to wait for each other in a \emph{dependency tree}, i.e., a node may have to wait for one parent node having switched to the new rule before being able to switch itself. Let the destination $d$ be the root of the dependency tree.

We will show two solutions, one with advantages regarding practice as the parent in the dependency forest is always a neighbor node. The practical solution is provably fast, however, it is not optimal regarding dependencies. So, in addition, we also show a more theoretical solution, which (in polynomial time) can compute the minimal dependency forest, i.e. where nodes can switch to the new solution in minimal time.
A node $p$ is a parent of a child node $c$ in the dependency forest if $c$ points to $p$ according to the new rules, e.g. $x$ is a parent of $u$ and $v$. Now, starting at the root, nodes first switch to the new rule, and then inform all their children in the dependency tree to switch as well. Apart from the packets-in-transit problem described above, correctness is immediate: Nodes that are in the dependency tree of the destination will only switch to the new rule once all the nodes on the new path to the destination have already switched. Based on the discussion on the example, it is easy to see that this solution is not minimal, as node $v$ cannot switch immediately, after learning the new rule, but must wait on $x$.


\subsection{Minimal Protocol}

In the following, we will present a protocol which is minimal regarding its dependencies. As the example shows, there may other nodes that can switch immediately. As such the dependency tree turns into a \emph{dependency forest}. As before, in this dependency forest, parents will inform their children when it is safe to use the new rule. The only difference is that the dependency forest is cut to its bare minimum. For simplicity, we first describe the protocol as if there was just a single destination $d$. We then discuss the case where we have multiple destinations.

We need to define \emph{good} and \emph{bad} nodes as follows: The destination $d$ is by definition good. All other nodes $u$ are defined as good or bad regarding the node $v$ at the end of the new forwarding rule \texttt{u.new = v}. If all paths (mixing new and [TODO: still valid] old rules) of $v$ point to the same node $r$ (possibly $r = d$), without loop, then $u$ is good. Else ($v$ has a path that points into a loop) $u$ is bad.

[TODO: is there a framework for algorithms? maybe not... text version instead]

The algorithm to construct the dependency forest is as follows: Initially, we let all good nodes to be roots of the dependency forest. Then, iteratively, while not all nodes are processed, we process an arbitrary good node $u$ by removing its old rule. Removing $u$'s old rule might turn other nodes good; if $v$ turns good when processing $u$, then $v$ is a child of $u$ in the dependency forest.

[TODO: here the old psudo-code]

Algorithm:
All good nodes u have no parent in the virtual forest, i.e. u.parent = nil, good nodes are ready to be processed.
While not all nodes are processed
	Process an arbitrary good node u: remove the old pointer u.old
	For all bad nodes v:
		If processing u made v good, then v.parent = u

[TODO: Remark: This also works with more than two trees, i.e. with multiple versions of old trees and one new tree. Should I describe a more direct algorithm, an algorithm that does recognize good and bad without calling the Tarjan loop discovery subroutine? What about the failures? In several versions, what happens if a node just does not ack the changes of old versions? Will it become a leaf?]

Lemmas:

Lemma 1: A processed node remains good.

Proof: After node $u$ is processed, $u$ inherits all the properties from its new parent \texttt{u.new = v}. TODO: why not losing the property again? Just removing pointers? What about interior collection points r?

Lemma 2: Every node is eventually processed.

Proof: Take any unprocessed node. We follow its new pointer until we find a pair $u,v$ with \texttt{u.new = v}, $u$ is not yet processed, and $v$ is processed. Since the root of the new pointer in-tree is already processed at the start of the algorithm, and we started at an unprocessed node, such a pair must exist. Because of Lemma 1, $v$ is good, and as such $u$ can be processed.

Lemma 3: The process produces a virtual forest.

Proof: Originally good nodes are the roots of the virtual forest. If a node $u$ is processed later than a node $v$, then $v$ cannot point to $u$. Listing all the nodes in the order of processing as such only gives parent pointers in one direction, towards the past. As such the virtual forest does not have cycles.

Lemma 4: The process is correct/sufficient (no loops).

Proof: For the sake of contradiction, assume that there is a packet which is sent into a loop. The loop must consist of at least one new pointer, call that \texttt{u.new}. In other words, node $u$ in the loop deleted its old pointer because it was good. However, we know that \texttt{u.new} is part of a loop, in other words, we know that the algorithm made a mistake when it decided that $u$ should be good (last line of algorithm). (TODO: simplify)

Lemma 5: The process is optimal/necessary/minimal (no node could flip earlier).

Proof: If nodes can flip immediately, they are by definition optimal. So let us look instead at child node $c$, which must wait for parent $p$, i.e. \texttt{c.par = p}. Node $c$ becomes good once parent $p$ removes its old pointer, before node $c$ was bad. This means that \texttt{p.old} pointed to a loop, which was removed when \texttt{p.old} was removed. In other words, by definition node $c$ could not switch to the new pointer at any earlier stage, without risking sending a packet on exactly this loop.





\begin{table*}[t!]
\begin{center}
\begin{small}
\begin{tabular}{>{\centering\arraybackslash}p{0.7in}|>{\centering\arraybackslash}m{0.75in}|>{\centering\arraybackslash}p{0.8in}|>{\centering\arraybackslash}p{0.85in}|>{\centering\arraybackslash}p{0.85in}|>{\centering\arraybackslash}p{0.75in}|}
&
  \textbf{None}
&
  \textbf{Self}
&
  \textbf{Downstream subset}
&
  \textbf{Downstream all}
&
  \textbf{Global}
\ \\ \hline

  \textbf{Eventual consistency}
&
  Always guaranteed
&
&
&
&
\ \\ \hline

  \textbf{Drop freedom}
&
  \multicolumn{1}{>{\columncolor[gray]{0.8}}c|}{Impossible}
&
  Add before remove
&
&
&
\ \\ \hline

  \textbf{Memory limit}
&
  \multicolumn{1}{>{\columncolor[gray]{0.8}}c|}{Impossible}
&
  Remove before add
&
&
&
\ \\ \hline

  \textbf{Loop freedom}
&
  \multicolumn{2}{>{\columncolor[gray]{0.8}}c|}{Impossible (Lemma \iflongversion \ref{lemma:imp loop-free} \else 6 \fi )}
&
  Rule dep. forest (\S\ref{sec:minimal})
&
  Rule dep. tree (\S\ref{sec:practical})
&
\ \\ \hline

  \textbf{Packet coherence}
&
  \multicolumn{3}{>{\columncolor[gray]{0.8}}c|}{Impossible (Lemma \iflongversion \ref{lemma:imp packet coherence} \else 7 \fi)}
&
  Per-flow ver. numbers
&
  Global ver. numbers~\cite{safeupdate}
\ \\ \hline

  \textbf{Bandwidth limit}
&
  \multicolumn{4}{>{\columncolor[gray]{0.8}}c|}{Impossible (Lemma  \iflongversion \ref{lemma:imp bandwidth limit} \else 8 \fi)}
&
  Staged partial moves~\cite{swan}
\ \\ \hline
\end{tabular}
\end{small}
\end{center}
\caption{Some basic consistency properties and their dependencies. Proofs of lemmas are in 
\iflongversion
the Appendix.
\else
~\cite{tr}.
\fi
}
\label{tbl:big}
\end{table*}

\section{Consistency space}
\label{sec:table}

Thus far, we have focused on one consistency property, to expose the various subtleties in maintaining consistency during updates. We now take a broader view of the range of consistency properties. Table~\ref{tbl:big} helps frame this discussion. Its rows correspond to consistency properties. We defined loop freedom and packet coherence in \S\ref{sec:loop-free}; the others are:

\paragraphb{Eventual} No consistency is provided during updates. If the new set of rules computed by the controller are consistent (by any definition), the network will be eventually consistent.

\paragraphb{Drop freedom} No packet should be dropped during updates. Drops may occur if a switch does not have a rule to handle a packet, and it is not configured to send unmatched packets to the controller (as is done in large-scale networks~\cite{swan,b4}).

%\item
%\textbf{Loop freedom} There should be no loops during updates, where we define a loop as a packet (without any transformation) visiting an (TODO: ``the same'' instead of ``an''?) interface multiple times.

%\paragraphb{Packet coherence} The set of rules seen by a packet should not be a mix of old and new rules; they should be either all old or all new rules.

\paragraphb{Memory limit} The number of rules that a switch is required to hold is always below a certain limit. A natural limit is the physical capacity of the flow table, but other limits may also be enforced.

\paragraphb{Bandwidth limit} The amount of traffic arriving at a link should not exceed a certain limit. As above, physical link capacity is a natural limit, but other limits may be interesting as well (e.g., margin for burstiness). Implicit in this definition is that the limit be maintained without dropping traffic; otherwise, we can meet any limit by dropping all traffic.

The consistency properties we list are not the only ones of interest.
%In some applications, one may be happy with eventual consistency, the weakest consistency property possible that just guarantees that eventually, all switches will be following the new rules, and then the network is by definition consistent again.
Some networks may require different properties (e.g., balanced load across two links), and some others may require  guarantees that combine two or more properties (e.g., packet coherence + bandwidth limits). We chose these consistency properties because they are basic and natural, as they capture the experience of packets and network elements.

The consistency properties are listed in rough order of strength, and satisfying a property lower on the list may satisfy a property above it. Obviously, packet coherence implies drop and loop freedom (assuming that the old and new rules sets are free of drops and loops). Perhaps less obviously, bandwidth limits imply loop freedom because sending even a small flow in a loop will quickly surpass any bandwidth limit for links on the loop.

However, these properties cannot be totally ordered. Packet coherence and bandwidth limits are orthogonal, as packet coherence does not address bandwidth, and bandwidth limits can be achieved with solutions beyond packet coherence.
%Roger: not much better than before, I know.
Drop freedom and loop freedom are also orthogonal. In fact, trivial solutions for one violates the other---dropping packets before they enter a loop guarantees loop freedom, and just sending packets back to the sender provides drop freedom but creates loops.

%As such we established all relations between the four consistency properties we list in the table.

The columns in Table~\ref{tbl:big} correspond to dependency structures. They denote which other switches must be updated before a new rule at a switch can be used (while maintaining consistency). Thus, the dependency is at the rule level, not switch level. (At the switch level, dependencies are often circular; a rule on switch $u$ depends on a rule on switch $v$, which in turn again depends on $u$ for other rules.)
%Roger: oops, I did it again.
Further, it captures when a rule can be installed and used, not when old rules can be removed. Even after all new rules are installed and the network is carrying traffic accordingly, the set of rules in the network may not be identical to the desired new rule set. Additional (unused, low-priority) rules may still exist in the network. Such rules will be removed in a clean-up phase;
%With the exception of drop freedom,
%Our dependency definition does not consider this phase but focuses on what it takes to get the network carrying traffic as per the new rules.
we will discuss this in Section \ref{sec:discussion}.

The different structures in Table~\ref{tbl:big} are:

\paragraphb{None} The rule does not depend on any other update.

\paragraphb{Self} The rule depends on updates at the same switch.

\paragraphb{Downstream subset} The update depends on updates at a subset of switches that lie downstream with respect to impacted packets.

\paragraphb{Downstream all} The update depends on updates at all switches that lie downstream with respect to impacted packets.

\paragraphb{Global} The rule depends on updates even at potentially all switches, including those that are not on the path for packets that use the rule.

%Some characteristics of these dependency properties are worth mentioning. First, note that the dependencies are totally ordered, i.e., no dependency is the weakest, and all is the strongest.

While these dependency structures are qualitative, not quantitative (e.g., time or length of dependency chains), in general, update procedures with fewer dependencies (i.e., to the left) are preferable.

%\ratul{I don't get this: In other words, the dependency categories are on a qualitative level only, and do not give the same insights as a more quantitative understanding on the level of rules. In SWAN \cite{swan}, for instance, progress towards the new solution is achieved in stages, and nodes need to wait with moving to the next stage until other nodes completed the last stage. The goal is to minimize the time until we can use a new solution.}

The cells in the table denote whether an algorithm exists to update the network while satisfying the corresponding consistency property and inducing the corresponding dependency structure. We see that certain combinations are impossible, that is, no such algorithm can exist.\footnote{These impossibility proofs are in Appendix \ref{sec:app2}.} So, for example, packet coherence cannot be achieved in a way that rules depend on updates at only a subset of downstream switches.

Expectedly, weaker consistency properties (towards the top) require weaker dependency structures (towards the left). At one extreme, eventual consistency, which implies no consistency during updates, requires no dependencies at all.  Slightly stronger properties, drop freedom and memory limit, can be achieved by depending on the switch itself. A simple algorithm for drop freedom is to add the new rule in the switch before the old rule is removed. When installed with higher priority, the new rules become immediately usable, without wait.
%\ratul{should drop freedom be none? there is no dependency on other rules, in terms of when a new rule can be added and used} %not sure that we need anything here, maybe something else that goes into 5?
A simplistic method for adhering to memory limits, if that were the only concern, is to remove some rules at the same switch before adding any new rule. This process might compromise other consistency properties (e.g., drop or loop freedom), but it will maintain memory limits.

At the other extreme, bandwidth limit requires global coordination. The intuition here is that maintaining bandwidth limits at a link requires coordinating all flows that use it, and some of these flows share links with other flows, and so on. Hong et al.~\cite{swan} describe a procedure to effect such transitions by moving flows partially across multiple stages.

Interestingly, all cells to the immediate right of impossible cells are occupied, which implies that, across past work and this paper,  (qualitatively) optimal algorithms for maintain all these consistency properties are known. However, one must not infer from this observation that finding consistent update procedures is a ``solved problem'' at this point.  As we mentioned previously, some networks may need different properties and algorithms will need to be developed for those properties.

Even amongst this set of properties, the rosy picture that emerges is partly due to the rows focusing on consistency properties in isolation. The combinations are hard to  ensure, and efficient algorithms are not known. For instance, drop freedom and memory limit, while easy to ensure individually, are challenging to ensure in combination. Maintaining the combination requires global dependencies, as introducing some rule at a switch might need to remove another rule first, which can only be removed after having added a new rule somewhere else.

Last not least, the table only shows the qualitative part of the story, ignoring the more interesting quantitative effects. Even though \cite{safeupdate} and \cite{swan} both have global dependencies, \cite{safeupdate} can always resolve the dependencies in only two rounds, whereas \cite{swan} may provably need more stages in certain instances. We believe that what is presented in this and prior work is just the tip of iceberg when it comes to consistent updates in SDNs.

\section{An architecture for SDN updates}
\label{sec:discussion}

In a conventional view of SDNs, the primary responsibility of the controller is to generate switch rules that are policy-compliant, but we have argued that maintaining consistency during rule updates is another key responsibility. The question is how this should be accomplished in a flexible, modular manner. One possibility is for the same software module to decide on new rules and then micro manage updates while ensuring consistency. However, this monolithic architecture is undesirable because it mixes three separable concerns--- $i)$ the network should route traffic per policy; $ii)$ the network should maintain consistency during updates; $iii)$ the updates should be efficient, which depends on the network characteristics (e.g., the mean and variance of applying an update to a switch).

\begin{figure}[t!]
  \centering
  \includegraphics[width=\columnwidth]{figures/arch.png}\\
  \caption{Proposed architecture}\label{fig:arch}
\end{figure}


%In general, software defined networks should not be modeled by mere rules or rule changes, but rather by \emph{rule dependencies}. As we showed in this paper, rules changes often depend on each other, before a new rule can be used, one must make sure that other rules are already in place, or that other rules are being removed. These rule dependencies in general form a directed graph we call the dependency graph, as follows: The nodes of the dependency graph are rule changes, e.g. insert, remove, or change a rule at some node. These nodes are connected by directed edges that represent dependencies between rule changes. If a node does not have an edge pointing to it, the rule can be implemented immediately.

We propose an alternative architecture, shown in Figure~\ref{fig:arch}. It is composed of three parts, corresponding to the three concerns above: $i)$ the {\em rule generator} produces new switch rules that comply with network policy; $ii)$ the {\em update plan generator} produces a plan for updating the network to the new rules in a way that ensures consistency per the desired property; and $iii)$  the {\em plan optimizer and executor} that is responsible for efficiently updating the network per the plan. In our architecture, the update plan is represented using an {\em update DAG} (directed acyclic graph). The nodes in this graph are updates (i.e., rule additions, deletions, or changes), and directed edges represent dependencies between updates. If an update does not have an edge pointing to it, it can be implemented immediately.

However, using only updates as nodes is not expressive enough. For example, in Reitblatt et al.'s procedure rules with old version numbers can be safely removed only after enough time has passed for any in-transit packets to drain~\cite{safeupdate}. As another example, some updates may have dependencies such that they can be applied when any one of their parents have been applied. To see this, switch the rules of node $u$ in Figure \ref{fig:minimal}, such that $u.old = w$ and $u.new = d$. Then, in order to prevent a loop, node $v$ must wait  for either $u$ \emph{or} $w$; it does not need to wait for both, and we shouldn't make it wait for both when applying updates.

%
%Depending on the application, one may add several special nodes to increase the functionality of the dependency graph. A general problem are slow packets in transit. For instance, the packet coherence protocol by Reitblatt et al.~\cite{safeupdate} may remove old rules while packets using these rules are in still transit. Concretely, consider a packet $p$ in transit from $y$ to $x$ in the example of Figure \ref{fig:example}. If node $x$ already passed the last step of the consistency protocol~\cite{safeupdate} and removed the old rule when receiving packet $p$, node $x$ has not other option than dropping packet $p$ once it arrives. A simple technique to deal with packets in transit is adding reasonable timers before removing old rules, i.e. by adding a node in the dependency graph that demands to ``wait 10 seconds'' that points to the node that removes the old rule.

%Also, sometimes nodes that incorporate logical functions may be helpful. As an example, switch the rules of node $u$ in Figure \ref{fig:minimal}, such that $u.old = w$ and $u.new = d$. Then, in order to prevent a loop, node $v$ must wait, but node $v$ may wait for either node $u$ \emph{or} node $w$. In the dependency graph, one may now choose to include either the dependency between $u$ and $v$, or the dependency between $w$ and $v$. Even though both are correct, the dependency graph allows a better solution, as we can just include an ``or'' node, connecting the insertions of the new rules at $u$ and $w$ with the new rule at $v$, and then $v$ just needs to wait for the faster of $u$ and $w$.

To efficiently and generally handle such dependencies, we introduce {\em combinator nodes} into the update DAG. Current combinators are delay and logical functions. A delay combinator is considered applied (i.e., its dependent updates can be safely applied) after the configured amount of time has elapsed since its parent updates were applied. A logical combinator is a logical function (e.g., AND or OR) over the binary state (applied or not) of its parents. It is considered applied when the function evaluates to true. These two combinators suffice to represent consistent update plans for all procedures in Table~\ref{tbl:big}, but future work may uncover the need for more types of combinators.

\paragraphb{The update plan generator} proceeds in two steps. It first computes, using the old rules, new rules, and the desired consistency property, a dependency graph where nodes correspond to deletion of old rules or addition of new rules. It then converts this graph into an update DAG such that, starting from the old rules, applying the DAG leads to new rules. This step is straightforward if the graph is cycle-free.

Otherwise, we must break cycles. How best to do this depends on the consistency property, but a few general tools help. One is using version numbers~\cite{safeupdate}, which particularly help if a new rule must wait for another old rule to be removed. Introducing version numbers makes it clear which of the (otherwise) conflicting rules should be used. As an extension, if one is not willing to wait for a rule to be inserted, additional information (similar to source routing) may be embedded in the packet as to how it should be handled downstream  and the circular dependency vanishes.

Another tool for breaking cycles is using {\em helper updates}, which exist in neither the old or new rule sets, but help with consistent transition from old to new rule sets. SWAN's staged partial moves~\cite{swan} can be expressed using helper updates that move subsets of flows per stage to avoid overloading any link. Circularity introduced due to prefix-based routing (\S\ref{sec:multidest}) can also be broken using helper updates. For example, in Figure \ref{fig:multidest} we can eliminate the cycle by breaking a single (default) rule into one for each of the two destinations covered by the default rule, introducing these rules during the update process, and then removing them later.

%
%However, the main reason for dependency graphs is that they allow for a general understanding of the consistency of SDN updates. First, as our prefix routing example in Section \label{ref:multidest} showed, dependencies might have cycles, and a dependency graph is a perfect means to understand these cycles in a formal way. One might decide to eliminate these cycles by simply removing dependency edges, with the advantage to completely understand that exactly these removed consistency properties will be violated by doing so. But one can also handle cycles without violating consistency. In the example in Figure \ref{fig:multidest}, for instance, one can eliminate the cycle by breaking a single (default) rule into one for each of the two destinations covered by the default rule. TODO: describe details if needed?

%Another technique to eliminate cycles is to introduce version numbers. Version numbers in particular help if a rule needs to wait for another rule to be removed first. Just introduce a version number, and then it is clear which of the (otherwise) conflicting rules should be used. What if one is not willing to wait for a rule to be inserted? Just include the additional information in the packet, similarly to source routing, and the dependency is vanishes.

\paragraphb{The plan executor} is responsible for applying the update DAG correctly and quickly. In deciding the order and timing of updates, it needs to factor in several concerns, including the delays from the controller to the switch, the time a switch takes to apply an update, and balancing load on individual switches. Further, in some cases, the update DAG may contain a long chain of dependencies. In the worst case, the chain can be of length $O(n)$, where $n$ is the number of nodes in the network,\footnote{For example, in a ring network, going from clockwise to counter-clockwise routing in a loop free manner.} and $O(n^2)$ single-hop messages will be needed because it can $O(n)$ such messages for a switch and controller to communicate. Such long chains may be shortened, for instance, using version numbers.  Alternatively, we may also introduce a new primitive in switches by which a switch informs its dependents when it is safe to apply an update, which would enable updates to be done in $O(n)$ messages. Developing efficient algorithms for applying update DAGs, while accounting for all these concerns, is a rich area for future work.

Sometimes, in the middle of forwarding state changes, we may need to change the network to yet another state (e.g., because we need to respond to failures). Such events are easier to handle in our architecture. The rule generator can compute the new state of the network, without worrying about the current, transient state. The plan executor knows the current state of the network, with some small uncertainty that corresponds to update messages that have been sent to the switches but have not been acknowledged. Using this state as the starting point, the plan generator can generate a new update DAG, which may cancel out updates in the old DAG. Then, the plan executor can start applying this new update DAG.


%The other main reason for dependency graphs is efficiency: Not only do they allow for a better understanding of minimal dependencies as shown in Section \ref{fig:minimal}. Also they provide a solid basis for optimizations. For instance, there are dependency graph that inevitably will contain a long chain of dependencies, and version numbers and other techniques help flattening such dependency graphs. For the sake of concreteness, consider a $n$-node network with a ring topology. In the old regime, all nodes point clockwise to destination $d$. In the new regime, all nodes point counter-clockwise to destination $d$. Let us call clockwise neighbor of $d$ node $u_1$, the clockwise neighbor of $u_i$ is node $u_{i+1}$. In other words, the counter-clockwise neighbor of $d$ is node $u_{n-1}$. In this example, the dependency graph for loop freedom is a linked list $u_1,u_2,\ldots,u_{n-1}$. If the dependency graph is synchronized with an SDN controller, no matter where in the ring network the SDN controller is located, $\Theta(n^2)$ messages are going to be transmitted before the network converged to the new solution. For loop freedom, this is the worst example, as the dependency graph for a single destination cannot be worse than a linked list. In such a case, it is possible to speed up conversion by using sequence numbers, or by having nodes informing neighbors directly, without going through the SDN controller. In both cases, the update can be performed using $O(n)$ messages only.

%So, the general procedure is as follows: First, the new rules are used to compute a dependency graph. This graph is then checked for cycles, which are eliminated using various techniques such as rule splitting or version numbers. Then the resulting cycle-free dependency graph (DAG) is fed into an optimizer, which speeds up conversion at the cost of introducing more memory, version numbers, direct update information between nodes, source routing, helper nodes, or other mechanisms. We end up with a flat dependency DAG, which can be directly implemented in the network.

%TODO: safeupdate is more parallel, however, using optimizations, dependencies do not have to be sequential, and can be optimized.

\section{Preliminary evaluation}
\label{sec:eval}

We have started implementing the architecture above. Our initial focus is on loop freedom and the update DAGs that emerge for it. In one experiment, we took Rocketfuel ISP topologies that were annotated with intra-domain routing weights~\cite{rocketfuel-weights}. We considered link failures in these topologies, and our goal was to update the network from pre-failure to post-failure least cost routing, while maintaining loop freedom.

Figure~\ref{fig:as} plots the distribution of the length of dependency chains that emerge for each topology across ten trials, where a single randomly selected link was failed in each, and update DAGs were computed using the minimal procedure in \S\ref{sec:minimal}. We see that roughly half of the updates depend on 0 or 1 other switch, and 90\% of all rules are dependent on at most 3 other switches. In contrast, had we used Reitblatt's procedure, which provides the stronger property of packet coherence, rules would have had to wait for all other switches (well over hundred in some topologies), and a single slow switch can impede everyone.

We also see a small fraction of cases with chain lengths greater than 5. These are prime candidates for implementation through localized chain shortening optimizations by the plan executor (and the plan generator can exclusively focus on maintaining loop freedom)


\begin{figure}[t!]
  \centering
  \includegraphics[width=2.5in]{figures/as.png}\\
  \caption{Chain lengths in update DAGs that emerge in response to link failures in six Rocketfuel topologies.}\label{fig:as}
\end{figure}

%Some additional snippets:
%
%The nodes in this graph are rules and they point to other rules that must be updated before them to guarantee loop freedom. While these graphs can be computed using an algorithm similar to the one above \ratul{true? need more details -- }, we cannot use the same update procedure that goes from roots to leafs. Instead, special care is needed for updating portions of the graph that form cycles.
%
%Nodes in a cycle can be updated in two ways. The first is using Reitblatt's procedure, which relies in version numbers for "atomically switching" a set of nodes. Another is to "break the cycles" by using helper rules that control subsets of the address space, thus removing the overlap. In the example above, \ratul{finish}.
%
%Which of the two methods is preferable depends on the context and other procedures may exist as well.
%
%Let's analyze it. There are only two reasons why we need network algorithms and protocols and the first place, first failures, second changes in either demand or infrastructure; if everything would be stable forever, there is no need to react (have protocols). In this work we essentially concentrate on the second (changes which can be planned). The first is also interesting, but mostly beyond the scope of this work. (This is where Ratul disagrees, and he is probably right.)
%
%Problem: Reaction time to failures and updates. A central controller is often slower than a distributed protocol, as it might be farther away, and the source of the problem/change is often close to those components that are most affected by a problem/change. Nevertheless, that’s again a failure problem. Is the only reason to use distributed protocols failure handling?
%
%(This is what this paper is about.) The answer is no. There is also synchronization. Even in the absence of failures, one cannot have nodes of a network migrate to a new version of operation at the very same instant. In theory, the last statement is trivial. In practice, clearly one tries to get as close as possible to optimal synchronization. Clearly, having an SDN helps a lot, as migration in heterogeneous networks is a whole different battle (this is another thing we discussed), as global updates of standard protocols show. However, even if the system is homogeneous, (time) synchronization is difficult [firing squad, or clock synchronization].
%
%Synchronization is an overloaded term, unfortunately. On the one hand we have something like (physical) clock synchronization, on the other hand we have (logical) synchronization. Perfect physical synchronization is impossible, so people often do logical synchronization. This is also what we do here or in SWAN.
%
%Or, to put it differently, as nodes cannot all migrate to the new version at exactly the same instance (and packets might be in transit still, and what not), we need a way to do a ``save transition'' from the old to the new state.
%


%First, what happens if the network is in the middle of a transition, and the SDN controller decides to update some rules (that might or might not have been rolled out already)? This can be integrated easily in the algorithm described above. In a nutshell, we just need to consider all rules that are possibly still existing in the network as old rules. As such, a node $u$ may have several old rules, pointing to different neighbors, plus one new rule. If some node $u$ was in limbo state before the new update was rolled out, it may have used either its old rule, or the intermediate rule (formerly known as new rule, now overwritten by the new rule). We know that mixing old and intermediate rules did not cause any loops, and as such we can treat both of them as old rules. If some rules have already been deleted, or have not been initiated, they can be ignored. Note that our algorithm also works in the presence of a whole set of old rules, and as such, it can automatically handle updates that interrupt transitions.

%A main cause for interrupting updates may be failures at nodes or edges. Essentially, failures may be handled in a standard way, by simply rolling out another batch of rules that will fix the failures. However, there is one exception, which we will first describe with an example: If some link $l$ is considered to be down, new rules will be introduced to route around this failed link $l$. An old rule on link $l$ might have a loop with some of the new rules, and as such the algorithm cannot push the new rules before the old rule is removed. Even worse, if the node $u$ holding the old rule is not accessible, the algorithm will not be able to push the fixing rules before node $u$ is reachable again. The SDN controller has a conflict of interest, it needs to quickly push a fix which must ignore (delete) the old rule, however, if unreachable node $u$ comes up again while doing so, the existing old rule will introduce a loop. (TODO: must be re-written, I just wanted to describe the problem.)


%Some additional snippets:
%
%Let’s analyze it. There are only two reasons why we need network algorithms and protocols and the first place, first failures, second changes in either demand or infrastructure; if everything would be stable forever, there is no need to react (have protocols). In this work we essentially concentrate on the second (changes which can be planned). The first is also interesting, but mostly beyond the scope of this work. (This is where Ratul disagrees, and he is probably right.)
%
%Problem: Reaction time to failures and updates. A central controller is often slower than a distributed protocol, as it might be farther away, and the source of the problem/change is often close to those components that are most affected by a problem/change. Nevertheless, that’s again a failure problem. Is the only reason to use distributed protocols failure handling?
%
%(This is what this paper is about.) The answer is no. There is also synchronization. Even in the absence of failures, one cannot have nodes of a network migrate to a new version of operation at the very same instant. In theory, the last statement is trivial. In practice, clearly one tries to get as close as possible to optimal synchronization. Clearly, having an SDN helps a lot, as migration in heterogeneous networks is a whole different battle (this is another thing we discussed), as global updates of standard protocols show. However, even if the system is homogeneous, (time) synchronization is difficult [firing squad, or clock synchronization].
%
%Synchronization is an overloaded term, unfortunately. On the one hand we have something like (physical) clock synchronization, on the other hand we have (logical) synchronization. Perfect physical synchronization is impossible, so people often do logical synchronization. This is also what we do here or in SWAN.
%
%Or, to put it differently, as nodes cannot all migrate to the new version at exactly the same instance (and packets might be in transit still, and what not), we need a way to do a ``save transition'' from the old to the new state.
%
%[Roger removed this missing snippet and used it in 4]
%
%Apart from time, there is also a space component, as we want to remove old rules as soon as possible (as soon as not used anymore).
%
%In this paper we discuss what the ``safe'' in safe transition is.
%
%We also discuss whether there is a tradeoff speed and safety.
%
%We exemplarily look into this tradeoff in one specific example (by looking at different ways to do such a transition).
%Some new ideas from discussions:
%
%What helper tools do you have? ``Just control network'' (including gateway routers, but not servers). If needed, one can add header, or mess with TTL. Also: Only talk to neighbors, or able to route messages?
%
%Other consistencies: FIFO, no drop, ``eventual'', suffix.
%
%Worst-case view: If there is a network/traffic pattern with a problem, then the consistency has a problem.
%
%You do not control old and new, so old=new is not a consistency criterion.
%
%

\section{Summary}
\label{sec:conclusions}

We argued that consistent updates in SDNs is an important and rich area for future research. We highlighted the trade-off between the strength of the consistency property and the dependency structure it imposes, and developed minimal algorithms for loop freedom. We also sketched an architecture for consistent and quick updates in SDNs.  

%{
%\scriptsize
%\vspace{5pt}
%\baselineskip=7.685bp
\bibliographystyle{abbrv}
\bibliography{paper}
%}
%\bibliographystyle{plain}

\iflongversion
\begin{appendix}
\section{Appendix for \S2.2}

In the following, we prove a few simple lemmas, which show that $i)$ the dependency forest is correct and $ii)$ minimal in the sense that if any node switches to the new rule before the parent in the dependency forest says so, a packet might be experiencing a loop.

\begin{lemma}\label{lemma:invariant} The invariant of the algorithm is that the current rules in the network are without loops.
\end{lemma}

\begin{proof} The invariant is true at the beginning, since no new rule is included, and the old rules form an in-tree to the destination $d$. The invariant remains true when a node enters the limbo state (uses both old and new rules) as we check at this stage that no loop is introduced when doing so. Finally, the invariant is true as well when a node enters the new state: As this only removes an old rule, no loop can be introduced.
\end{proof}

\begin{lemma}\label{lemma:loop-free} The dependency forest is loop-free.
\end{lemma}

\begin{proof} As the parent of a child must be in the dependency forest already when the child enters the dependency forest, there cannot be loops in the dependency forest.
\end{proof}

\begin{lemma}\label{lemma:forest} At the end, every node (but the destination $d$) is in the dependency forest.
\end{lemma}

\begin{proof} As long as there are nodes in limbo state, we will move them to the new state, one after the other, possibly moving other nodes from old to limbo. What if no node remains in limbo state, i.e. all nodes are either new or old? Then, as the algorithm suggests, there is always at least one node which can directly jump from old to new. We can find this node as follows: Start from an arbitrary old node, and move along the new rules towards the destination $d$. Since the destination is (by definition) new, along this new-rules path, there must be a last pair of nodes $c,p$, where $c.new = p$, $c$ and $p$ are old and new, respectively. Node $c$ can directly move to state new, $c$'s parent in the dependency forest is $p$. Removing $c$'s old rule will not introduce a loop, as removing a rule never introduces loops. Also, adding $c$'s new rule does not introduce a loop, as $c.new$ points to nodes which are in the new state already, that is, there are no more old rules which can cause loops.
\end{proof}

Note that there are networks where this jumping mechanism is necessary. Example: A network with three nodes, where $u.old = v$, $u.new = d$, $v.old = d$, and $v.new = u$. Either of the nodes $u,v$ cannot enter the limbo state, as there will be a loop between $u$ and $v$.

\begin{lemma}\label{lemma:correctness} The process is correct (and produces no loops).
\end{lemma}

\begin{proof} The follows directly from Lemmas \ref{lemma:invariant}, \ref{lemma:loop-free} and \ref{lemma:forest}.
\end{proof}

\begin{lemma}\label{lemma:minimal} The process is minimal.
\end{lemma}

\begin{proof}
Root nodes in the dependency forest can flip to the new rule immediately, and are as such by definition optimal. A node $c$ is a child of a parent node $p$ in the dependency forest, exactly because $c$ can only use its new rule after $p$ removed its old rule. As such, the process guaranteed that $c$ was added to the dependency tree at the earliest possible (minimal) time.
\end{proof}

\section{Appendix for \S3}
\label{sec:app2}


\begin{lemma}\label{lemma:imp loop-free} Loop freedom depends on other nodes.
\end{lemma}

\begin{proof}
In the example in Figure \ref{fig:example}, node $x$ depends on node $y$.
\end{proof}

\begin{lemma}\label{lemma:imp packet coherence} Packet coherence depends on all non-trivial downstream switches.
\end{lemma}

\begin{proof} Let $u$ be a switch router that is non-trivial, in the sense that $u$ is affected by a rule change, i.e. $u$'s old rule differs from its new rule. If the source starts to route packets according to the new rule, the switch $u$ will forward the packets wrongly, or drop them, which is not packet coherent.
\end{proof}

\begin{lemma}\label{lemma:imp bandwidth limit} Bandwidth limit potentially depends on all switches.
\end{lemma}

\begin{proof} Let $f$ be a flow that wants to use a new path $p$, or increase its capacity on an existing path. The network may be able to adapt to flow $f$, however, in general only if other flows use different paths as well, which in turn may (recursively) move even other flows (some of which have no single switch/link in common with the new path $p$). As such, any $f$ may potentially depend on any single switch in the network.
\end{proof}

Note that, in some networks and applications, one may be able to implement bandwidth limit with less dependencies.

\end{appendix}
\fi

\end{sloppypar}
\end{document}


