\section{Loop-Freedom}
\label{sec:loop-free}

\subsection{Example}
\label{sec:example}

\begin{figure}[t!]
\includegraphics[width=3in]{figures/noloops.png}
\caption{Running example, discussed in Section \ref{sec:example}.}
\label{fig:example}
\end{figure}

[TODO: why do these figures look so crummy in my printout?]

Applying routing updates in a way that guarantees no loops is a basic but already interesting consistency property. Consider the five node example network with a single destination $d$ in Figure \ref{fig:example}. The existing routing to destination $d$ on the left of the figure should be updated according to the right of the figure. A naive way to update is to simply send out the forwarding updates (e.g., ``node $v$: for destination $d$ send to $x$''). However, if nodes just apply these updates, it might happen that node $x$ updates its rule before node $y$, introducing a routing loop between nodes $x$ and $y$. This loop will eventually disappear, namely once node $y$ updates its rule, but in an asynchronous system it is difficult to give guarantees when this will happen.

At the other end of the spectrum, the SDN controller could first (i) distribute the new rules, then (ii) wait for an acknowledgement of all the nodes that they have received the new rules, then (iii) tell all the nodes to stop sending packets, and finally, once (iv) they all acknowledged that, (v) tell the nodes to now use the new rules. After the nodes (vi) acknowledge that they are using the new rules, the SDN controller can tell them to (vii) remove the old rules, as they are not needed anymore. This solution does not suffer from loops, as the old and the new solution are well-separated in time.\footnote{Strictly speaking, this protocol is not sufficient, as there might be packets in transit, still using the old rules when rushing through the protocol. Consider a packet sent from sent from $y$ to $x$ shortly before $y$ received the order to stop sending (iii). If node $x$ already passed the last step of the protocol when receiving the packet, node $x$ has not other option then sending the packet back to node $y$, i.e. the packet is experiencing a loop.} However, it is also terribly slow. One may speed up the process by omitting steps (iii) and (iv). Now, assuming that node $x$ received the command to use the new rule (v) earlier than node $y$. As such, in order to guarantee no loop between nodes $x$ and $y$, we must introduce version numbers in packets such that nodes $x$ and $y$ know that packets from $x$ respectively $y$ must be treated according to the new respectively old rules. This is the solution proposed by \cite{theoriginalprincetonpaper}.

One may ask whether version numbers are really necessary, just to guarantee no loops? Also, one may ask whether a faster solution is possible. What nodes are really dependent on each other? This may look like a technicality, but as we all know, nodes are often temporarily unavailable or reacting slowly. If a node must wait for another node, there should better be a consistency reason for it, and not merely protocol overhead. Is there something like a \emph{minimal protocol for a given consistency guarantee}? This is exactly the question we address in this paper.

Regarding our example, the answer to this question is quite simple. Node $u$ does not even need to change its rule, as it always just forwards to $x$, hence node $u$ does not even have to be informed about the rule changes. Node $v$ can switch immediately after being informed about the new rule, as no matter whether using the old or new rule, the packet will always end up at node $x$, with no possibility to experience a loop. Also node $y$ can switch immediately to the new rule, as its packet will then directly reach the destination node $d$. The only critical node in our example is node $x$, for which we understand that node $x$ must wait until node $y$ implemented the switch, as otherwise our network might experience a loop.

More generally, achieving loop-free consistency requires rule updates to wait for each other in a \emph{dependency tree}, i.e., a node may have to wait for one parent node having switched to the new rule before being able to switch itself. A simple dependency tree works as follows: Let the destination $d$ be the root of the dependency tree. A node $c$ is a child of parent $p$ in the dependency tree if and only if the new rule of node $c$ points to node $p$. In our example in Figure \ref{fig:example}, $x$ is a parent of $u$ and $v$. 

Now, starting at the root of the dependency tree, nodes first switch to the new rule, and then inform all their children in the dependency tree to switch as well. Apart from the packets-in-transit problem described in footnote 1 (TODO), correctness is immediate: Nodes that are in the dependency tree of the destination will only switch to the new rule once all the nodes on the new path to the destination have already switched. However, based on the discussion on the example, it is easy to see that this solution is not minimal, as node $v$ cannot switch immediately, after learning the new rule, but must wait on $x$.



\begin{figure}[t!]
\includegraphics[width=3in]{figures/nominimum.png}
\caption{Example with two orthogonal minimal solutions, discussed in Section \ref{sec:minimal}.}
\label{fig:minimal}
\end{figure}

In Section \label{sec:minimal}, we will show a slightly more complicated algorithm which produces a minimal result that can also be computed in polynomial time.  One may wish for a protocol which is as fast as possible, in the sense that dependencies are minimum. Quite
surprisingly, in general this is not trivial. Consider the four node example network in Figure \ref{fig:minimal}, again with the old rules on the left, and the new rules on the right. Node $w$ may switch to the new rule immediately. However, nodes $u$ and $v$ may not. If they both switch to the new rule immediately, and $w$ is still using the old rule, we introduced a loop. So one of them must wait for $w$ having switched. However, either one is fine, i.e. either $u$ must wait for $w$ (and $v,w$ may switch immediately), \emph{or} $v$ must wait for $w$ (and $u,w$ may switch immediately). In other words, there are two dependency trees, one which is fast for node $u$ (at the expense of node $v$), and one which is fast for node $v$ (at the expense of node $u$). Clearly, one cannot argue that one solution is better than the other, as the two solutions are orthogonal. As such, this example shows that \emph{the} minimum solution does not exist, and we must look for \emph{a} minimal solution instead.

Moreover, if we switch the rules of node $u$ in the example in Figure \ref{fig:minimal}, such that \texttt{u.old = w} and \texttt{u.new = d}, we have another interesting case: Now, in order to prevent a loop, node $v$ must wait, but node $v$ may wait for either node $u$ \emph{or} node $w$. These examples show that optimality is a non-trivial concept that deserves further research.

\subsection{Minimal Protocol}
\label{sec:minimal}

Instead, in the following, we will present a protocol which is only \emph{minimal} regarding its dependencies, in the sense that no node can improve its dependencies without some other node getting worse. As the example in Figure \ref{fig:example} shows, often there may be many nodes that can switch immediately, for different reasons. As such the dependency tree turns into a \emph{dependency forest}. As before, in this dependency forest, parents will inform their children when it is safe to use the new rule. The only difference is that the dependency forest is minimal, in the sense that one cannot remove a subtree of the forest with subtree root $c$ (and parent $p$) and re-attach it at a parent of $p$, or make $c$ a root of the forest. For simplicity, we first describe the protocol as if there was just a single destination $d$. We later discuss the case where we have multiple destinations.

Each node is in one of three states, \emph{old}, \emph{new}, or \emph{limbo}, depending on whether the node is guaranteed to only use its old rule or new rule, or whether it might use both rules, respectively. Since the destination $d$ does not have any rules, neither old nor new, it is by definition in state new.

The algorithm constructs the dependency forest as follows: We start out with only the old rules, by definition a loop-free in-tree to destination $d$. Now, for each node $u$, we test whether adding $u$'s new rule will introduce a loop. If not, node $u$ is entering the state limbo, and added as a root in the dependency forest. On the other hand, if $u$'s new rule introduces a loop, node $u$ remains old, and must wait until we find its parent in the dependency forest. The initialization is completed after we found all the roots (which are now leaves in the dependency forest). Next we add the children to the dependency forest, one after another. In each step, we choose a limbo (dependency forest leaf) node $u$. We remove $u$'s old rule from the network (putting $u$ in the new state), and then cycle through all the old nodes, trying to find nodes $v$ where the new rule of $v$ does not introduce a loop, thanks to removing $u$'s old rule. If we find such a node $v$, then node $v$ is a child of $u$ in the dependency forest, and a new leaf of the dependency forest, in limbo state. If no such node $v$ exists, node $u$ remains a childless leaf of the dependency forest. It may happen that we find no more limbo node, that is, all nodes are either new or old. In this case, as proved below in Lemma 3 (TODO), we will always find an old node that can become new directly.
We continue until we processed all nodes in the network.

At the heart of this algorithm is the simple loop-finder of Tarjan \cite{reference_1_in_http://en.wikipedia.org/wiki/Tarjan's_strongly_connected_components_algorithm} which can be implemented linearly in the size of the network. Since we must process each node, and check for each not yet processed node, the total time of the algorithm is cubic in the size of the network. We suspect that this running time can be improved, but this is beyond the scope of this paper. Clearly, the order in which the nodes are visited will determine the dependency forest, however, we will guarantee that no matter what the order is, the dependency tree is minimal.

TODO: alternative formulation, shorter, maybe better: Initially, we let all good nodes to be roots of the dependency forest. Then, iteratively, while not all nodes are processed, we process an arbitrary good node $u$ by removing its old rule. Removing $u$'s old rule might turn other nodes good; if $v$ turns good when processing $u$, then $v$ is a child of $u$ in the dependency forest.

TODO: explain Tarjan in simple terms.

\subsection{Proofs} %TODO: should probably not be a subsection, just did that for navigability.

In the following, we prove a few simple lemmas, which show that (i) the dependency forest is correct and (ii) minimal in the sense that if any node switches to the new rule before the parent in the dependency forest says so, a packet might be experiencing a loop.

Lemma 1: The invariant of the algorithm is that the current rules in the network are without loops.

Proof: The invariant is true at the beginning, since no new rule is included, and the old rules form an in-tree to the destination $d$. The invariant remains true when a node enters the limbo state (uses both old and new rules) as we check at this stage that no loop is introduced when doing so. Finally, the statement is true as well when a node enters the new state: As this only removes an old rule, no loop can be introduced.

Lemma 2: The dependency forest is loop-free.

Proof: As the parent of a child must be in the dependency forest already, there cannot be loops in the dependency forest, when the child enters the dependency forest.

Lemma 3: At the end, every node (but the destination $d$) is in the dependency forest.

Proof: As long as there are nodes in limbo state, we will move them to the new state, one after the other, possibly moving other nodes from old to limbo. What if no node remains in limbo state, i.e. all nodes are either new or old. Then, as the algorithm suggests, there is always at least one node which can directly jump from old state to new state. We can find this node as follows: Start from an arbitrary old node, and move along the new rules towards the destination $d$. Since the destination is (by definition) new, along this new-rules path, there must be a last pair of nodes $c,p$, where \texttt{c.new = p}, $c$ and $p$ are old and new, respectively. Node $c$ can directly move to state new, $c$'s parent in the dependency forest is $p$. Removing $c$'s old rule will not introduce a loop, as removing a rule never introduces new loops. Also, adding $c$'s new rule does not introduce a loop, as it points to nodes which are in the new state already, that is, there are no more old rules which can cause loops.

Remark: There are networks where this jumping mechanism is necessary. Example: A network with three nodes, where \texttt{u.old = v}, \texttt{u.new = d}, \texttt{v.old = d}, and \texttt{v.new = u}. Either of the nodes $u,v$ cannot enter the limbo state, as there will be a loop between $u$ and $v$.

Lemma 4: The process is correct (and produces no loops).

Proof: The follows directly from the dependency tree, as we only added nodes once they did not introduce a loop (Lemma 1).

Lemma 5: The process is minimal.

Proof: Root nodes in the dependency forest can flip to the new rule immediately, and are as such by definition optimal. A node $c$ is a child of a parent node $p$ in the dependency tree, exactly because $c$ can only use its new rule after $p$ removed its old rule. As such, the process guaranteed that $c$ was added to the dependency tree at the earliest possible time.

\subsection{Discussion} %TODO: should probably not be a subsection, just did that for navigability.

Let us now discuss some of the issues and additional features of this minimal dependency forest. First, what happens if the network is in the middle of a transition, and the SDN controller decides to update some rules (that might or might not have been rolled out already)? This can be integrated easily in the algorithm described above. In a nutshell, we just need to consider all rules that are possibly still existing in the network as old rules. As such, a node $u$ may have several old rules, pointing to different neighbors, plus one new rule. If some node $u$ was in limbo state before the new update was rolled out, it may have used either its old rule, or the intermediate rule (formerly known as new rule, now overwritten by the new rule). We know that mixing old and intermediate rules did not cause any loops, and as such we can treat both of them as old rules. If some rules have already been deleted, or have not been initiated, they can be ignored. Note that our algorithm also works in the presence of a whole set of old rules, and as such, it can automatically handle updates that interrupt transitions.

A main cause for interrupting updates may be failures at nodes or edges. Essentially, failures may be handled in a standard way, by simply rolling out another batch of rules that will fix the failures. However, there is one exception, which we will first describe with an example: If some link $l$ is considered to be down, new rules will be introduced to route around this failed link $l$. An old rule on link $l$ might have a loop with some of the new rules, and as such the algorithm cannot push the new rules before the old rule is removed. Even worse, if the node $u$ holding the old rule is not accessible, the algorithm will not be able to push the fixing rules before node $u$ is reachable again. The SDN controller has a conflict of interest, it needs to quickly push a fix which must ignore (delete) the old rule, however, if unreachable node $u$ comes up again while doing so, the existing old rule will introduce a loop. (TODO: must be re-written, I just wanted to describe the problem.)

It is generally debatable whether SDN controllers are the best way to quickly handle errors. Going through the SDN controller always introduces some extra lag, after detecting an error. One may want to consider additional distributed ways to fix an error quickly, e.g. techniques such as link reversal \cite{originallinkreversalpaperforinstanceorsomethingnewer}

One other question is regarding time, in particular, how bad is the worst-case updates. Consider a $n$-node network with a ring topology. In the old regime, all nodes point clockwise to destination $d$. In the new regime, all nodes point counter-clockwise to destination $d$. Let us call clockwise neighbor of $d$ node $u_1$, the clockwise neighbor of $u_i$ is node $u_{i+1}$. In other words, the counter-clockwise neighbor of $d$ is node $u_{n-1}$. In this example, the dependency forest is a linked list $u_1,u_2,\ldots,u_{n-1}$, in other words, if the dependency forest is synchronized with a SDN controller, no matter where in the ring network the SDN controller is located, $\Theta(n^2)$ messages are going to be exchanged before the network adopted the new solution. This is the worst example, as the dependency forest cannot be worse than a linked list.

One other major question is what if we update several routes to several (or even all) destinations at the same time. A new rule at a node $u$ may be responsible for a whole prefix $p$ of destinations. In this case, optimality/minimality may be defined in different ways, as some sub-prefixes of prefix $p$ at node $u$ may be good earlier than others. For example, it may be that prefix $p0$ will be good immediately, as it remains unchanged, whereas prefix $p1$ will end up in a loop if we switch immediately. One may now say that the new rule for prefix $p$ should be split up into two rules $p0$ and $p1$, making it possible to switch at least $p0$ immediately. Then the solution above is straight-forward, as we just split up the addressing space into all the necessary minimal sub-prefixes, computing dependency forests for all of them independently. However, even though this solution is in some sense time-minimal, it may be space-inefficient since new rules may now be split up into sub-prefix rules.

Instead, one may want to define minimality with the original new rules in mind, that is, the new rule of node $u$ for prefix $p$ should not be used before it is loop-free. However, one may end up with a deadlock. The canonical example is a triangle network with three nodes and destinations $u,v,w$. The old routing is always clockwise, for two hops. The new routing is counter-clockwise, also for two hops. In other words, in both the old and the new solution, on each of the three links, two destinations are bundled into one \emph{default} rule. Now, no matter which of the new counter-clockwise rules is initiated first, we immediately generate a loop for one destination. This discussion shows that there is an interesting trade-off between time and space when looking at concurrent changes for multiple destinations.
