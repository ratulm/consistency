\section{Introduction}
\label{sec:intro}

Software-defined networking (SDN), as envisioned today in separation of control and data planes and centralized control plane computation, is a relatively new concept.
%receiving a lot of attention in the research and practitioner communities.
From early papers (e.g.,~\cite{rcp-case,rcp,4d,ethane}), we learn that its primary promises were $i)$ centralized control plane computation can eliminate the ill-effects (e.g., looping packets) of distributed computation, and $ii)$ separation of control and data planes enables flexible configuration of the data plane in a manner that satisfies a wide range of policy concerns.
For example, the Routing Control Platform (RCP)~\cite{rcp-case,rcp} was motivated by ``oscillations" and ``loops" that can occur in certain iBGP architectures. To avoid these problems, it advocated a centralized control plane that directly configured the data plane of routers in an autonomous system.

4D aimed to simplify network management~\cite{4d}. It observed that the ``data plane needs to implement, in addition to next-hop forwarding, functions such as tunneling, access control, address translation, and queuing." In today's networks, this ``requires complex arrangements of commands to tag routes, filter routes, and configure multiple interacting routing processes, all the while ensuring that no router is asked to handle more routes and packet filters than it has resources to cope with. A change to any one part of the configuration can easily break other parts." Based on this observation, it argues for centrally computing data plane state in way that obeys all concerns.

Similarly, ETHANE argued that for simplified management of enterprise networks ``policy should determine the path that packets follow"~\cite{ethane}. It then argued for SDNs because the requirements of network management ``are complex and require strong consistency, making it quite hard to compute in a distributed manner." These promises have led to SDNs garnering a lot of attention from the researchers as well as practitioners.

However, as we gain more experience with this paradigm, a nuanced story is emerging.  Researchers have shown that, even with SDNs, packets can take paths that do not comply with policy~\cite{safeupdate,xx} and that more traffic can arrive at a link than its capacity~\cite{swan}. So, what explains this gap between the promise and these inconsistencies. The root cause is that promises apply to the eventual behavior of the network, {\em after} data plane state has been changed and the inconsistencies emerge {\em during} data plane state changes.

Since changes to data plane state, in response to failures, load changes, and policy changes, are an essential part of an operational network, so will be the inconsistencies. Thus, successfully using SDNs means that we need not only methods for computing policy-compliant data plane state but also methods to manage state changes in a way that maintains desired consistency properties.

Beyond the specific contradictory behaviors and solutions studies by prior work~\cite{safeupdate,swan}, we present a holistic analysis of such behaviors and what is fundamentally needed to prevent them.

We do not mean to imply that SDNs are not a net positive towards tackling the problems they aim to tackle. We merely wish to highlight that the story (i.e., use centralized computation) is not that straightforward.


5.	First, we outline a spectrum of consistency properties:
a.	Drop-freedom
b.	Loop-freedom
c.	Per-pkt consistency
d.	Capacity consistency
e.	…..

6.	Consistently migrating requires a careful planning of the order in which rules can be installed in the network. This means a dependence in which certain rules should be applied only after certain other rules have been applied.  Based on the observation, we outline a spectrum of the types of possible dependencies:
a.	None
b.	Downstream of the path
c.	Entire network

7.	Greater dependence slows transition. But certain consistency properties inherently require a minimum degree of dependence. We fill out the table …

8.	We propose new transition algorithms along the way
