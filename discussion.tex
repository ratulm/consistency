\section{Discussion}
\label{sec:discussion}

\subsection{A general abstraction of network updates}

<<<<<<< HEAD
\paragraph{Dependency primitive in switches}

\paragraph{Dependency graphs}

In general, software defined networks should not be modeled by mere rules or rule changes, but rather by \emph{rule dependencies}. As we showed in this paper, rules often depend on each other, before a new rule can be used, one must make sure that other rules are already in place, or that other rules are being removed. These rule dependencies in general form a directed graph we call the dependency graph, as follows:

TODO: Not clear how to describe the graph. Do nodes do something, or are they just there? It is clear that edges will carry the dependencies (rule exists, or rule is removed). Edges could also carry actions (insert rule, delete rule), but that could possibly also be done by nodes. What is better? What if we have logical functions in the graph, e.g. a logical or? This can also be represented. What about usage? It is clear that usage depends on certain rules to be present. However, also the opposite is true, e.g. dropping a rule may depend on the rule no longer being used.

There are two basic problems with dependency graphs. First, they may contain cycles, e.g. the triangle example for loop-free routing with multiple destinations. These cycles must be broken, and there are several ways to do this: One my for instance add helper nodes into the graph that for instance break conflicting rules into subrules. TODO: one may go through the triangle example to explain this. Another technique to break cycles is to introduce version numbers. Version numbers in particular help if a rule needs to wait for another rule to be removed first. Just introduce a version number, and then it is clear which of the conflicting rules should be used. 

There is also a helping scheme in order not to wait for another rule to be in place already: Also here we include additional information in the packet, namely the information what the node that did not install the necessary rule yet should do with the packet when it receives it (and does not know what to do with it). 

Version numbers also help flattening the dependency graph. With version numbers, one may be able to break long dependency chains, and the ``add all new rules first, using a new version number, and once that is done, use them'' paradigm shortens dependencies in a simple way.




Some additional snippets:

Let’s analyze it. There are only two reasons why we need network algorithms and protocols and the first place, first failures, second changes in either demand or infrastructure; if everything would be stable forever, there is no need to react (have protocols). In this work we essentially concentrate on the second (changes which can be planned). The first is also interesting, but mostly beyond the scope of this work. (This is where Ratul disagrees, and he is probably right.)

Problem: Reaction time to failures and updates. A central controller is often slower than a distributed protocol, as it might be farther away, and the source of the problem/change is often close to those components that are most affected by a problem/change. Nevertheless, that’s again a failure problem. Is the only reason to use distributed protocols failure handling?

(This is what this paper is about.) The answer is no. There is also synchronization. Even in the absence of failures, one cannot have nodes of a network migrate to a new version of operation at the very same instant. In theory, the last statement is trivial. In practice, clearly one tries to get as close as possible to optimal synchronization. Clearly, having an SDN helps a lot, as migration in heterogeneous networks is a whole different battle (this is another thing we discussed), as global updates of standard protocols show. However, even if the system is homogeneous, (time) synchronization is difficult [firing squad, or clock synchronization].

Synchronization is an overloaded term, unfortunately. On the one hand we have something like (physical) clock synchronization, on the other hand we have (logical) synchronization. Perfect physical synchronization is impossible, so people often do logical synchronization. This is also what we do here or in SWAN.

Or, to put it differently, as nodes cannot all migrate to the new version at exactly the same instance (and packets might be in transit still, and what not), we need a way to do a ``save transition'' from the old to the new state.
=======
\paragraph{Dependencies vs. time} It is clear that less dependencies improve an update process, as slow or unreachable nodes will slow down less less nodes. However, if all nodes respond timely, less dependence may actually slow down the update process.
>>>>>>> b3c7e0fc742ea021afcbaebc9fcc9f0348e30c5e

One other question is regarding time, in particular, how bad is the worst-case updates. Consider a $n$-node network with a ring topology. In the old regime, all nodes point clockwise to destination $d$. In the new regime, all nodes point counter-clockwise to destination $d$. Let us call clockwise neighbor of $d$ node $u_1$, the clockwise neighbor of $u_i$ is node $u_{i+1}$. In other words, the counter-clockwise neighbor of $d$ is node $u_{n-1}$. In this example, the dependency forest is a linked list $u_1,u_2,\ldots,u_{n-1}$, in other words, if the dependency forest is synchronized with a SDN controller, no matter where in the ring network the SDN controller is located, $\Theta(n^2)$ messages are going to be exchanged before the network adopted the new solution. This is the worst example, as the dependency forest cannot be worse than a linked list.

\paragraph{Dependency primitive in switches}

\paragraph{Dependency graphs}

First, what happens if the network is in the middle of a transition, and the SDN controller decides to update some rules (that might or might not have been rolled out already)? This can be integrated easily in the algorithm described above. In a nutshell, we just need to consider all rules that are possibly still existing in the network as old rules. As such, a node $u$ may have several old rules, pointing to different neighbors, plus one new rule. If some node $u$ was in limbo state before the new update was rolled out, it may have used either its old rule, or the intermediate rule (formerly known as new rule, now overwritten by the new rule). We know that mixing old and intermediate rules did not cause any loops, and as such we can treat both of them as old rules. If some rules have already been deleted, or have not been initiated, they can be ignored. Note that our algorithm also works in the presence of a whole set of old rules, and as such, it can automatically handle updates that interrupt transitions.

A main cause for interrupting updates may be failures at nodes or edges. Essentially, failures may be handled in a standard way, by simply rolling out another batch of rules that will fix the failures. However, there is one exception, which we will first describe with an example: If some link $l$ is considered to be down, new rules will be introduced to route around this failed link $l$. An old rule on link $l$ might have a loop with some of the new rules, and as such the algorithm cannot push the new rules before the old rule is removed. Even worse, if the node $u$ holding the old rule is not accessible, the algorithm will not be able to push the fixing rules before node $u$ is reachable again. The SDN controller has a conflict of interest, it needs to quickly push a fix which must ignore (delete) the old rule, however, if unreachable node $u$ comes up again while doing so, the existing old rule will introduce a loop. (TODO: must be re-written, I just wanted to describe the problem.)


%Some additional snippets:
%
%Let’s analyze it. There are only two reasons why we need network algorithms and protocols and the first place, first failures, second changes in either demand or infrastructure; if everything would be stable forever, there is no need to react (have protocols). In this work we essentially concentrate on the second (changes which can be planned). The first is also interesting, but mostly beyond the scope of this work. (This is where Ratul disagrees, and he is probably right.)
%
%Problem: Reaction time to failures and updates. A central controller is often slower than a distributed protocol, as it might be farther away, and the source of the problem/change is often close to those components that are most affected by a problem/change. Nevertheless, that’s again a failure problem. Is the only reason to use distributed protocols failure handling?
%
%(This is what this paper is about.) The answer is no. There is also synchronization. Even in the absence of failures, one cannot have nodes of a network migrate to a new version of operation at the very same instant. In theory, the last statement is trivial. In practice, clearly one tries to get as close as possible to optimal synchronization. Clearly, having an SDN helps a lot, as migration in heterogeneous networks is a whole different battle (this is another thing we discussed), as global updates of standard protocols show. However, even if the system is homogeneous, (time) synchronization is difficult [firing squad, or clock synchronization].
%
%Synchronization is an overloaded term, unfortunately. On the one hand we have something like (physical) clock synchronization, on the other hand we have (logical) synchronization. Perfect physical synchronization is impossible, so people often do logical synchronization. This is also what we do here or in SWAN.
%
%Or, to put it differently, as nodes cannot all migrate to the new version at exactly the same instance (and packets might be in transit still, and what not), we need a way to do a ``save transition'' from the old to the new state.
%
%[Roger removed this missing snippet and used it in 4]
%
%Apart from time, there is also a space component, as we want to remove old rules as soon as possible (as soon as not used anymore).
%
%In this paper we discuss what the ``safe'' in safe transition is.
%
%We also discuss whether there is a tradeoff speed and safety.
%
%We exemplarily look into this tradeoff in one specific example (by looking at different ways to do such a transition).
%Some new ideas from discussions:
%
%What helper tools do you have? ``Just control network'' (including gateway routers, but not servers). If needed, one can add header, or mess with TTL. Also: Only talk to neighbors, or able to route messages?
%
%Other consistencies: FIFO, no drop, ``eventual'', suffix.
%
%Worst-case view: If there is a network/traffic pattern with a problem, then the consistency has a problem.
%
%You do not control old and new, so old=new is not a consistency criterion.
%
%
